\documentclass{beamer}
\usepackage[utf8]{inputenc}
\usepackage[english]{babel}
\usepackage[T1]{fontenc}
\usepackage{appendixnumberbeamer}
\usepackage{minted}
\setminted{linenos, numberblanklines=false, mathescape, texcomments, autogobble, breakanywhere, breakautoindent, breaklines, frame=lines, framesep=5\fboxsep}
\setminted[python]{python3}

\usefonttheme{professionalfonts}
\usepackage{mathspec}
\setsansfont[BoldFont={Fira Sans},
Numbers={OldStyle}]{Fira Sans Light}
\setmathsfont(Digits)[Numbers={Lining, Proportional}]{Fira Sans Light}

\def\mytitle{Title}
\def\mysubtitle{Subtitle}
\def\myshortfullname{A. Falai}
\def\myfullname{Alessio Falai}
\def\myemail{alessio.falai@studio.unibo.it}
\def\myshortinstitute{UNIBO}
\def\myinstitute{Alma Mater Studiorum - University of Bologna}

\title{\mytitle}
\subtitle{\mysubtitle}
\author[\myshortfullname]{\myfullname \newline \texttt{\myemail}}
\date{\today}
\institute[\myshortinstitute]{\myinstitute}
\titlegraphic{\hfill\includegraphics[height=1.5cm]{img/unibo_logo.eps}}

\usetheme[progressbar=frametitle, numbering=none, background=light, titleformat=smallcaps, block=fill]{metropolis}
\useoutertheme{metropolis}
\useinnertheme{metropolis}
\usefonttheme{metropolis}
\usecolortheme{metropolis}
\setbeamercovered{dynamic}

\begin{document}
\begin{frame}[noframenumbering]
	\maketitle
\end{frame}

\begin{frame}[noframenumbering]
	\frametitle{Table of contents}
	\setbeamertemplate{section in toc}[sections numbered]
	\tableofcontents[hideallsubsections]
\end{frame}

\makeatother
\setbeamertemplate{footline}
{
  \leavevmode%
  \hbox{%
  \begin{beamercolorbox}[wd=.3\paperwidth,ht=2.25ex,dp=1ex,center]{author in head/foot}%
    \usebeamerfont{author in head/foot}\insertshortauthor ~ (\myshortinstitute)
  \end{beamercolorbox}%
  \begin{beamercolorbox}[wd=.6\paperwidth,ht=2.25ex,dp=1ex,center]{block title}%
    \usebeamerfont{title in head/foot}\insertshorttitle
  \end{beamercolorbox}%
  \begin{beamercolorbox}[wd=.1\paperwidth,ht=2.25ex,dp=1ex,center]{block body}%
    \insertframenumber{} / \inserttotalframenumber\hspace*{1ex}
  \end{beamercolorbox}}%
  \vskip0pt%
}
\makeatletter
\setbeamertemplate{navigation symbols}{}

\section{Math stuff}
\begin{frame}
	\frametitle{Fermat's last theorem}
	\begin{theorem}
		The equation
		$$
		x^n+y^n=z^n
		$$
		has no integer solutions for $n > 2$ where $x, y, z \neq 0$.
	\end{theorem}
	\pause
	\begin{proof}
		The proof is trivial and left as an exercise for the reader.
	\end{proof}
\end{frame}

\section{Code stuff}
\begin{frame}[fragile]{Python code}
	\begin{minted}[label=Hello world]{python}
		def hello_world():
			print("Hello world")
	\end{minted}
\end{frame}

\begin{frame}[standout]
	Thank you for your attention
\end{frame}

\appendix
\begin{frame}
	\frametitle{References}
	\nocite{*}
    \bibliography{bibliography}
    \bibliographystyle{plain}
\end{frame}

\begin{frame}
	\frametitle{Backup frame}
	\usebeamercolor[fg]{normal text}
	This is a backup frame, useful to include additional material for questions from the audience.
\end{frame}
\end{document}
