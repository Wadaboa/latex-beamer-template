\documentclass{beamer}
\usepackage[english]{babel}
\usepackage[utf8]{inputenc}
\usepackage[T1]{fontenc}
\title{Title}
\author[A. Falai]{Alessio Falai \newline \texttt{alessio.falai@studio.unibo.it}}
\date[01/01/1970]{01 January 1970}
\institute[UNIBO]{Alma Mater Studiorum - University of Bologna}
\logo{\includegraphics[width=15mm]{img/unibo_logo.eps}}
\usetheme{AnnArbor}
\useoutertheme[right]{sidebar}
\setbeamercovered{dynamic}

\begin{document}
\begin{frame}
	\maketitle
\end{frame}

\begin{frame}
	\frametitle{Table of contents}
	\tableofcontents
\end{frame}

\section{Introduction}
\begin{frame}
	\frametitle{Che cosa sono i numeri primi?}
	\begin{definition}
		Un \alert{numero primo} un intero $>1$ che ha esattamente
		due divisori positivi.
	\end{definition}
\end{frame}

\section{L’infinit dei primi}
\begin{frame}
	\frametitle{I numeri primi sono infiniti}
	\framesubtitle{Ne diamo una dimostrazione diretta}
	\begin{theorem}
		Non esiste un primo maggiore di tutti gli altri.
	\end{theorem}
	\pause
	\begin{proof}
		\begin{enumerate}[<+->]
			\item Sia dato un elenco di primi.
			\item Sia $q$ il loro prodotto.
			\item Allora $q+1$  divisibile per un primo $p$
			che non compare nellelenco. \qedhere
		\end{enumerate}
	\end{proof}
\end{frame}

\section{Problemi aperti}
\begin{frame}
	\frametitle{Che cosa c ancora da fare?}
	\begin{block}{Problemi risolti}
		Quanti sono i numeri primi?
	\end{block}
	\begin{block}{Problemi aperti}
		Un numero pari $>2$ sempre la somma di due primi?
	\end{block}
\end{frame}
\end{document}
